\documentclass[11pt]{article}
\usepackage{graphicx}
\usepackage{amsmath, amsthm, amssymb}
\usepackage{epsfig}
\usepackage{natbib}
\usepackage{cite}
\usepackage{url}
\usepackage[
  bookmarks=true,
  bookmarksnumbered=true,
  colorlinks=true,
  filecolor=blue,
  linkcolor=blue,
  urlcolor=blue,
  hyperfootnotes=true
  citecolor=blue
]{hyperref}
\begin{document}

\title[COMAP reduce docs]{\bf COMAP Memo 0: COMAP Manchester Reduction Pipeline Documentation}
\author{Stuart~Harper}
\date{March 5, 2019}
\setlength{\topmargin}{-15mm}

\maketitle

\tableofcontents
\newpage

\section{Installation}

Requirements:
\begin{itemize}
\item \textsc{Python} 3.0 or higher.
\item Compiled shared libraries of the \texttt{FORTRAN} version of the Starlink astronomical libraries (SLALIB) available from \url{http://starlink.eao.hawaii.edu/starlink/2018ADownload}.
\item A copy of parallel ready H5Py. Installation instructions can be found here \url{http://docs.h5py.org/en/stable/mpi.html}. N.B.: If you are using an Anaconda packaged version of \textsc{Python} installation you may need to remove the existing \textsc{conda} install of HDF5.
\item The latest version of \textsc{HealPy} , \textsc{mpi4py} (either openMPI or MPICH work as backends), \textsc{numpy}, \textsc{scipy}, \textsc{matplotlib}, and \textsc{astropy}.
\end{itemize}

To install the Manchester COMAP reduction pipeline:
\begin{itemize}
  \item Clone/download the \texttt{github} repository found here: \url{https://github.com/SharperJBCA/COMAPreduce}.
  \item Enter the directory: \textit{cd COMAPreduce} and run \textit{python setup.py install}. If your SLALIB libaries are not in standard location you must define the environment variable: $\texttt{SLALIB\_LIBS}$
  \item To run the COMAP pipeline make a new directory above COMAPreduce (e.g. \textit{cd ../ \&\& mkdir runcomapreduce}) and copy the \textsc{run.py}, and \textsc{\*.ini} files there.
  \item The pipeline can then be run using the command: \textit{mpirun -n X python run.py -F FILELIST.list -P PARAMETERS.ini}. \textit{FILELIST.list} should contain a list of files with either just the filenames to be processed or the full path to files to be processed. \textit{PARAMETERS.ini} will control the processing to be performed, details of which are described in Sections~\ref{sec:usage} and \ref{sec:classes}.
\end{itemize}

\section{Usage}\label{sec:usage}

\subsection{Parameter Files}

There are several example parameter files already included:
\begin{itemize}
  \item \textsc{AmbLoad.ini}    - This will calculate the $T_\mathrm{sys}$ and gain (e.g. volts per Kelvin) from ambient load stare observations.
  \item \textsc{Downsample.ini} - This will downsample a data file in frequency by \texttt{factor} times and also check to see if any pointing needs to be added.
  \item \textsc{FitJupiter.ini} - This will fix the pointing, downsample, and calibrate a Jupiter observation to the ambient load. Then it will fit a Gaussian to the time ordered data to derive amplitude, pointing and beam width measurements. It will also produce a calibration scale in units of Janskys/Kelvin for every horn and frequency channel.
\end{itemize}


\section{Classes}\label{sec:classes}

\subsection{BaseClass.H5Data}

Useful functions for defining new classes:
\begin{itemize}
  \item getdset - Retrieve a dataset, if it is not in memory load it.
  \item setdset - Load a dataset into memory.
  \item resizedset - Resize a dataset by passing it a new array.
  \item updatedset - Update a dataset values.
  \item getAttr    - Get an attribute (stored in output file)
  \item setAttr    - Set an attribute
  \item getextra - Get a dataset from the extra outputs
  \item setextra - Set an array to be assigned to extra outputs (must describe the shape of the array, e.g. which axis refers to horns, frequencies, etc... )
  \item resizeextra - Change dimensions of an extra dataset.
\end{itemize}
N.B. Never write directly to the dset or extras attributes of the H5Data class.

Useful attributes/functions for MPI routines:
\begin{itemize}
\item splitType - Axis type being split for MPI purposes (i.e., either Types.\_HORNS\_, Types.\_SIDEBANDS\_, Types.\_FREQUENCY\_, Types.\_TIME\_).
\item selectType - Axis type being explicity selected (i.e., as above)
\item selectIndex - Index being selected along selectType axis.
\item splitFields - Names of fields in COMAP data structure that will contain a split axis.
\item selectFields - Names of fields in COMAP data structure that will have a selected axis.
\item hi/lo - Dictionary, for each splitField, defining where in the larger structure this process is accessing data.
\item ndims - Dictionary containing dimensions of each dataset in memory for this process.
\item fullFieldLengths - Dictionary containing dimensions of each dataset in memory in total.
\item getDataRange - Function that returns how to split N values between M processes.
\end{itemize}



\end{document}
